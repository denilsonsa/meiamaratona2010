\problema{(Fra��es Equivalentes)}

\begin{descricao}
Um professor da disciplina Estudos de Racionais e Irracionais, ERI, d� aula sobre fra��es para seus alunos. A pr�xima aula ser� sobre fra��es equivalentes, e ele gostaria de ter um programa para os alunos brincarem com os n�meros. Como ele n�o sabe programar, pediu que voc� fizesse um programa que decidisse se duas fra��es s�o equivalentes. Para ajudar nesta tarefa, ele lhe deu uma dica de como verificar se duas fra��es s�o equivalentes. Dizemos que $\frac{x}{y}$ e $\frac{z}{w}$ s�o equivalentes se $x\times w = y\times z$.
\end{descricao}

\begin{entrada}
A primeira linha da entrada cont�m um inteiro positivo $N$ ($N \leq 1000$). Cada uma das $N$ linhas seguintes cont�m 4 n�meros inteiros positivos, $x$, $y$, $z$ e $w$, todos maiores ou iguais a $1$ e menores ou iguais a $100$.
\end{entrada}

\begin{saida}
Para cada qu�drupla $x$, $y$, $z$, $w$, voc� deve imprimir uma linha contendo ``SIM'' caso as fra��es sejam equivalentes e ``NAO'' (sem acento) em caso contr�rio.
\end{saida}

\exemplos{0.47}{0.47}
{3		\pl
1 2 3 4		\pl
1 2 2 4		\pl
50 50 100 100
}
{NAO	\pl
SIM	\pl
SIM
}

%\pb
