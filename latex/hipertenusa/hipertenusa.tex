\problema{(Hipertenusa)}

\begin{descricao}
Joe N�sio atualmente cursa a disciplina Estudando Retas e Inteiros. Nessa disciplina, s�o abordadas as rela��es entre n�meros inteiros e a geometria inerente a eles. Na primeira aula foram estudados tri�ngulos pitag�ricos, tri�ngulos ret�ngulos que possuem seus tr�s lados, os dois catetos e a hipotenusa, como n�meros inteiros. Al�m disso, o professor Pablo Top�zio mostrou que n�meros que s�o hipotenusas de algum tri�ngulo ret�ngulo satisfazem uma s�rie de propriedades.

Joe, muito interessado e criativo, se fez a seguinte pergunta: quais seriam os n�meros inteiros que s�o o quadrado da hipotenusa de pelo menos dois tri�ngulos ret�ngulos distintos com catetos inteiros? Ele batizou um n�mero que satisfaz essa propriedade de hipertenusa. Por exemplo, $50$ � uma hipertenusa, uma vez que $50 = 1^2 + 7^2 = 5^2 + 5^2$.

Interessado em obter uma resposta para essa pergunta, ele conversou com Hiago Bota e P�rola Silvia, seus companheiros de time da maratona de triangula��o (uma competi��o ex�tica onde times de tr�s pessoas devem resolver problemas sobre tri�ngulos em uma determinada quantidade de tempo), e pediu-os para ajud�-lo a escrever um programa respondesse ao questionamento. Ao descobrir que n�o sabem resolver esse problema e receosos que algo parecido acabe aparecendo na pr�xima maratona de triangula��o, eles pediram que voc� os ajudasse.
\end{descricao}

\begin{entrada}
A entrada � composta por diversas linhas de entrada. A primeira linha cont�m um inteiro $T \leq 1000000$, o n�mero de casos de teste. Cada uma das $T$ linhas seguintes cont�m um n�mero $1 \leq N \leq 10000000$, o n�mero que deve ser checado se � uma hipertenusa ou n�o.
\end{entrada}

\begin{saida}
Para cada linha da entrada deve ser impresso ``sim'', caso o n�mero seja uma hipertenusa e ``nao'' caso contr�rio.
\end{saida}

\exemplos{0.47}{0.47}
{4	\pl
37	\pl
43	\pl
50	\pl
51
}
{nao	\pl
nao	\pl
sim	\pl
nao
}

%\pb
