\problema{(Faces e Arestas)}

\begin{descricao}
No evento Esculturas Reconhecidas Internacionalmente, v�rios artistas enviam seus projetos de escultura para serem constru�dos em tamanho grande e apreciados por todos os visitantes.

Para que isso seja poss�vel, s�o impostas algumas limita��es �s esculturas aceitas no evento: 
Todas as esculturas podem ser constru�das a partir de pol�gonos planos (tri�ngulos, quadrados, etc).
As esculturas ser�o constru�das a partir de uma estrutura de arame, e sobre essa estrutura ser�o fixadas as placas planas nos formatos dos pol�gonos.
Todas as esculturas s�o ``fechadas'', de modo que n�o seja poss�vel ver a estrutura interna delas.

Para o pr�ximo evento, todas as esculturas que ser�o expostas j� foram submetidas, restando apenas encomendar o material necess�rio para constru�-las. Esse material � formado por placas cortadas no formato dos pol�gonos desejados e por barras retas que formam a estrutura interna de arame.

Pedro, respons�vel pelas encomendas, recebeu a lista com todas as placas necess�rias para cada escultura, por�m perdeu a lista que continha a quantidade de barras necess�rias para cada escultura. Em vez de requisitar outra lista (o que pode demorar alguns dias), ele pensou se poderia reconstruir a lista de barras a partir da lista de placas.
\end{descricao}

\begin{entrada}
A entrada � composta por v�rios casos de teste, cada um correspondendo a uma escultura. A primeira linha de cada caso de teste cont�m um n�mero inteiro $n$ ($1 \leq n \leq 100$), que indica quantos tipos diferentes de pol�gonos ser�o encomendados. As $n$ linhas seguintes cont�m, cada uma, dois n�meros inteiros $a$, $b$ ($2 \leq a \leq 100$, $1 \leq b \leq 1000$), que indicam respectivamente a quantidade de lados do pol�gono e quantos pol�gonos desse tipo s�o necess�rios.

A entrada termina com uma linha com $n=0$, que n�o deve ser processada.
\end{entrada}

\begin{saida}
Para cada caso de teste deve ser impressa uma �nica linha, contendo o n�mero de barras met�licas necess�rias para construir a estrutura de arame da escultura.
\end{saida}

\exemplos{0.47}{0.47}
{4	\pl
3 1	\pl
4 23	\pl
10 2	\pl
9 1	\pl
1	\pl
4 6	\pl
0
}
{62	\pl
12
}

%\pb
